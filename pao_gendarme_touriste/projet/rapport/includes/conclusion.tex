	Pour conclure, ce projet d’approfondissement et d'ouverture nous a permis d'apprendre à concevoir et développer une application Android. Nous avons découvert, à travers ce projet, de nouvelles technologies et avons pu acquérir de nouvelles expériences en nous initiant à des concepts qui nous étaient jusqu'alors inconnus.
	\\
	
	 Ensuite le développement de cette application nous aura permis de fournir un travail pour un client avec de réelles attentes. Ce client n'étant pas initié au domaine de la programmation, ce projet nous aura permis de nous familiariser avec la création d'un outils simplifié et accessible. De plus, le fait que le client soit la gendarmerie nous a inspiré une certaine motivation supplémentaire.
	 \\
	
	Concernant les perspectives de cette application, un premier point qui pourrait être amélioré concerne la liste des contacts téléphoniques. En effet, il peut être intéressant de rendre la construction de cette liste dynamique à l'image de celle des conseils et mettre en place la possibilité d'ajouter des numéros. Ensuite, un autre point très important concerne la version minimale d'Android avec laquelle l'application est compatible. Par exemple, dans notre cas la barre de navigation générale n'est pas compatible avec la version 2.3 d'Android c'est pourquoi il serait intéressant de trouver une alternative à cette dernière. Enfin, l'application sera testée en situation réelle par le client durant les mois de juin, juillet et août. Celui-ci nous fera part de ses remarques et des différents bugs rencontrés que nous pourrons alors étudier plus en profondeur.